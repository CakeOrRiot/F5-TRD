\documentclass[a4paper]{article}
\usepackage{listings}
%\usepackage{verbatim}
%\usepackage{scontents}
%\usepackage{fancyvrb}
\usepackage{graphicx}
\usepackage{sverb}
\usepackage[english,russian]{babel}
\usepackage{lastpage}
\usepackage[utf8]{inputenc}
\usepackage[T2A]{fontenc}
\usepackage{fancyhdr}
\usepackage[left=1cm,right=2cm,
    top=2cm,bottom=2cm,bindingoffset=0cm]{geometry}
\newcommand{\AddSection}[1]{\textbf{#1} \verbinput{#1}}

\pagestyle{fancy}
\fancyhf{}
\fancyhead[R]{Page \thepage\ of \pageref{LastPage}}
\fancyhead[L]{Ural Federal University (Nikolaev, Pozdnyakov, Sobolev)}
\begin{document}


\AddSection{For practice session.txt}

\AddSection{Template.txt}
\AddSection{Python file in out.txt}
\section{Number theory} 
\AddSection{Extended euclid.txt}
\AddSection{Modulo inverse.txt}
\AddSection{All modulo inverses.txt}
\AddSection{BigInt.txt}
\AddSection{Primes.txt}
\AddSection{Highly composite numbers.txt}

\section{Data structures}

\AddSection{Default segment tree.txt}
\AddSection{Segment tree with range updates.txt}
\AddSection{DSU.txt}


\section{Geometry}
\AddSection{Geometry basics.txt}
\AddSection{Convex polygon area.txt}
\AddSection{Convex hull.txt}

\section{Graph}
\AddSection{Dijkstra.txt}
\AddSection{Floyd-Warshall.txt}
\AddSection{Ford-Bellman.txt}
\AddSection{Strongly connected components.txt}
\section{Game Theory}
\AddSection{Game theory approach.txt}


\section{Strings}
\AddSection{Prefix function.txt}
\AddSection{Manaker.txt}
\AddSection{Suffix automata.txt}

\section{Other}
\AddSection{ternary search.txt}

\newpage

\begin{table}[]
\resizebox{\textwidth}{!}{%
\begin{tabular}{|l|l|l|l|l|}
\hline
  & Comment & М & С & А \\ \hline
A &         &   &   &   \\ \hline
B &         &   &   &   \\ \hline
C &         &   &   &   \\ \hline
D &         &   &   &   \\ \hline
E &         &   &   &   \\ \hline
F &         &   &   &   \\ \hline
G &         &   &   &   \\ \hline
H &         &   &   &   \\ \hline
I &         &   &   &   \\ \hline
J &         &   &   &   \\ \hline
K &         &   &   &   \\ \hline
L &         &   &   &   \\ \hline
M &         &   &   &   \\ \hline
\end{tabular}%
}
\end{table}

\end{document}
